\documentclass{article}
\usepackage{polski}
\usepackage[T1]{fontenc}
\usepackage[utf8x]{inputenc}
\usepackage{booktabs}
\usepackage{multirow}
\usepackage{graphicx}
\usepackage{textcomp}
\usepackage{eurosym}
\usepackage{float}
\usepackage{adjustbox}
\usepackage{graphics}
\usepackage{tabularx}
\usepackage{rotating}
\usepackage{tabulary}
\usepackage{listings}
\usepackage{amstext}
\usepackage{xcolor}
\usepackage{url,textcomp}
\usepackage{amssymb}


\title{SCR- sieci komputerowe - Laboratorium - Raport 1}

\author{Damian Ryś}

 
\begin{document}

\maketitle


\tableofcontents
\section{Wstęp}
\hspace*{0.5cm}Grupa lab: E12-93c	//TODO

Termin zajęć: CZW 15:15-16:55

Numer indeksu: 252936

Prowadzący: Dr inż. Jerzy Greblicki 
\section{Podstawowe komendy systemu Linux}
Do jedynych z najczęściej używanych poleceń podczas pracy z systemem Linux jest:
\begin{itemize}
    \item id - identyfikacja użytkownika
    \item passwd - zmiana hasła użytkownika
    \item who - informacje dotyczące bieżacej sesji
    \item cd - zmiana bieżącego katalogu na katalog będący argumentem polecenia
    \item mkdir - utworzenie nowego katalogu
    \item ls - wyświetlenie zawartości katalogu
    \item chmod - zmiana praw dostępu do pliku lub katalogu
    \item adduser - dodanie nowego użytkownika
\end{itemize}

\section{Instalacja Vim'a}
Zaczynamy od zaaktualizowania naszych pakietów przy użyciu
komendy \textit{sudo apt update}

\begin{figure}[H]
    \centering
    \hspace*{-1cm}
    \includegraphics[totalheight=10cm]{zlozonosc.jpg}
    \caption{Złożoność obliczeniowa dla metody enqueue}
    \label{2}
\end{figure}

Dopiero teraz jesteśmy w stanie poprawnie zainstalować edytor
tekstu przy użyciu komendy \textit{sudo apt install vim}

\begin{figure}[H]
    \centering
    \hspace*{-1cm}
    \includegraphics[totalheight=10cm]{zlozonosc.jpg}
    \caption{Złożoność obliczeniowa dla metody enqueue}
    \label{2}
\end{figure}


\section{dodac uzytkownika}

\begin{figure}[H]
    \centering
    \hspace*{-1cm}
    \includegraphics[totalheight=10cm]{zlozonosc.jpg}
    \caption{Złożoność obliczeniowa dla metody enqueue}
    \label{2}
\end{figure}


\section{Utworzenie nowej grupy - "studenci"}

\begin{figure}[H]
    \centering
    \hspace*{-1cm}
    \includegraphics[totalheight=10cm]{zlozonosc.jpg}
    \caption{Złożoność obliczeniowa dla metody enqueue}
    \label{2}
\end{figure}


\section{Komenda ls -la}

\begin{figure}[H]
    \centering
    \hspace*{-1cm}
    \includegraphics[totalheight=10cm]{zlozonosc.jpg}
    \caption{Złożoność obliczeniowa dla metody enqueue}
    \label{2}
\end{figure}


\section{Zmienianie uprawnienień użytkowników "chmod" }

\begin{figure}[H]
    \centering
    \hspace*{-1cm}
    \includegraphics[totalheight=10cm]{zlozonosc.jpg}
    \caption{Złożoność obliczeniowa dla metody enqueue}
    \label{2}
\end{figure}




\end{document}